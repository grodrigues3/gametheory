\documentclass{article}
\usepackage{graphicx,fancyhdr,amsmath,amssymb,amsthm,subfig,url,hyperref,bm}
\usepackage[margin=1in]{geometry}

%----------------------- Macros and Definitions --------------------------

%%% FILL THIS OUT
\newcommand{\studentname}{Garrett Rodrigues}
\newcommand{\exerciseset}{Exercise Set 2}
%%% END



\renewcommand{\theenumi}{\bf \arabic{enumi}}

%\theoremstyle{plain}
%\newtheorem{theorem}{Theorem}
%\newtheorem{lemma}[theorem]{Lemma}

\fancypagestyle{plain}{}
\pagestyle{fancy}
\fancyhf{}
\fancyhead[RO,LE]{\sffamily\bfseries\large Stanford University}
\fancyhead[LO,RE]{\sffamily\bfseries\large CS 364A Algorithmic Game Theory}
\fancyfoot[LO,RE]{\sffamily\bfseries\large \studentname}
\fancyfoot[RO,LE]{\sffamily\bfseries\thepage}
\renewcommand{\headrulewidth}{1pt}
\renewcommand{\footrulewidth}{1pt}

\graphicspath{{figures/}}

%-------------------------------- Title ----------------------------------

\title{CS364A \exerciseset}
\author{\studentname}

%--------------------------------- Text ----------------------------------

\begin{document}
\maketitle

\section*{Exercise 9}
Use Myerson’s Lemma to prove that the Vickrey auction is the unique single-item auction that is DSIC,
always awards the good to the highest bidder, and charges losers 0.

Recall the three components of Myserson's Lemma (for single parameter
environments).
\begin{enumerate}
\item An allocation rule x is implementable if and only if it is monotone
\item If x is monotone, then there is a unique payment rule such that the sealed-bid mechanism (x, p) is DSIC [assuming the normalization that bi = 0 implies pi(b) = 0]. 
\item The payment rule in (b) is given by an explicit formula 
\end{enumerate}

\begin{equation}
	p(b_i, \bm{b_{-i}}) = \sum_{j=1}^{l}z_j\text{jump in x}_i(\cdot,
	\bm{b_{-i}})\text{at $z_j$}
\end{equation}

Also recall that a Vickrey auction is defined as a sealed-bid auction where
the highest bidder wins and pays a price equal to the second highest bid.

Therefore, by definition, the Vickrey awards the good to the highest bidder
By Myerson's lemma, we know that for single parameter
enviroments (e.g. this single item auction), if x is monotone, then it is
implementable and there is a unique payment rule such that the sealed bid
mechanism is DSIC. In the Vickrey auction, the allocation rule looks like a
step function, with all bidders getting 0 allocation until we reach the
second highest bidder, after which the allocation is 1. Therefore, the
allocation rule in a Vickrey auction is monotone.

Let $b_{high}$ be the highest bidder and $\bm{b_{-high}}$ be all other
bidders.  Given the payment rule outlined in Myserson's lemma, the payment for
$\bm{b_{-high}}$ is 0 because there are no jumps before the highest bidder
(the only jump in the allocation happens AT the second highest bid).  For
$b_{high}$, the payment rule gives the prices as $z_j$ multiplied by the jump
in x.  $z_j$ is equal to the second highest bid and the jump in x at the
second highest bid is from 0 to 1.  Therefore the payment for the highest
bidder is equal to the second highest bid, exactly as outlined in the Vickrey
auction. 
\section*{Exercise 10}
Use the “payment difference sandwich” in the proof of Myerson’s Lemma to prove that if an allocation rule is not monotone, then it is not implementable.

In the payment sandwich, we defined two bids y and z with y $<$ z.  For those
two bids, we were able to use the "swapping trick" to come up with the payment
sandwich.  If x is not monotone, that means for some value of y and z (with y
smaller than z), there is an allocation such that $x(y) > x(z)$,
which implies that $x(y) - x(z) > 0$.  So from the payment sandwich, we get
that $y[x(y) - x(z)] \geq p(y) - p(z) \geq z[x(y) - x(z)]$.  Since we know
that $x(y) - x(z)$ is positive, a non-monotone allocation rule implies that y
must be $\geq$ z, which contradicts our definition that y $<$ z.
\section*{Exercise 11}
We concluded the proof of Myerson’s Lemma by giving a “proof by picture” that coupling a monotone and
piecewise constant allocation rule x with the payment formula (see above), yields a DSIC
mechanism. Where does the proof-by-picture break down if the piecewise constant allocation rule x is not
monotone?

If thi piecewise constant allocation rule x is not monotone, that means there
is some point at which there is a "downward" jump when you increase your bid.
That is, there is some point where a bidder's bid increases at which their
allocation decreases.  That means they could get more by bidding less and
their utility is no longer maximized by bidding truthfully.  By bidding less,
they would pay the same amount (since a drop in the allocation doesn't
contribute to an increased payment according the payment rule) and they would
get more surplus.  Since utility is just surplus - payment, if the surplus
increases and the payment doesn't, that means the utility increases when they
underbid, which breaks DSIC.

\section*{Exercise 12}
Give a purely algebraic proof that coupling a monotone and piecewise constant allocation rule x with the
payment rule (1) yields a DSIC mechanism.

To show this, we need to show that the Utility for a bidder in an auction
with a piecewise constant and montone allocation rule x and the payment rule
above is maximized when the bidder bids, their true value $v_i$.  To do this,
we will compare the utility of the bid $v_i$ to the utility of the bids
$b_{low}$ and $b_{high}$ and show that that utility of $v_i$ is guaranteed to
be $\geq$ than utility for a higher or lower bid.

\begin{enumerate}
	\item Recall that the utility for a bidder is given by $u(b) = v_i x(b) -
		p(b)$ 
	\item Therefore the utility of bidding $v_i$ is $u(v_i, \mathbf{b_{-i})} =
		v_i x(v_i) - p(v_i)$
	\item $u(v_i) - u(b_{high}) = [v_i x(v_i) - p(v_i)] - [v_i x(b_{high}) -
		p(b_{high})]$
	\item $u(v_i) - u(b_{high}) = v_i[x(v_i) - x(b_{high}] - p(v_i) + p(b_{high})$
\item $u(v_i) - u(b_{high}) = v_i[x(v_i) - x(b_{high}] +
	\sum_{k=1}^{l}z_k\text{jump in x}_i(\cdot, \bm{b_{-i}}) \text{at }z_k $
\item Where $z_1, z_2...z_k$ are the breakpoints in the function $x_i(\cdot,
	\bm{b_{-i}})$ in the range $[v_i, b_{high}]$
\item Because we know that $z_1, z_2,...z_k$ are all $\geq v_i$, we can change
	the equality to:
\item $u(v_i) - u(b_{high}) \geq v_i[x(v_i) - x(b_{high}] + v_i
	\sum_{k=1}^{l}\text{jump in x}_i(\cdot, \bm{b_{-i}})$ 
\item $u(v_i) - u(b_{high}) \geq v_i[x(v_i) - x(b_{high})] + v_i [x(b_{high}) -
	v_i]$
\item $u(v_i) - u(b_{high}) \geq 0$
\end{enumerate}

We can do something similar for a lower bid $b_{low}$

\begin{enumerate}
\item $u(v_i) - u(b_{low}) = [v_i x(v_i) - p(v_i)]] - [v_i x(b_{low}) -
		p(b_{low})$
\item $u(v_i) - u(b_{low}) = v_i[x(v_i) - x(b_{low}] - p(v_i) + p(b_{low})$
\item $u(v_i) - u(b_{low}) = v_i[x(v_i) - x(b_{low}] -
	\sum_{k=1}^{l}z_k\text{jump in x}_i(\cdot, \bm{b_{-i}}) \text{at }z_k $
\item Where $z_1, z_2...z_k$ are the breakpoints in the function $x_i(/cdot,
	\bm{b_{-i}})$ in the range $[b_{low}, v_i]$
\item Because we know that $z_1, z_2,...z_k$ are all $\leq v_i$, we can change
	the equality to (note the negative in from of the summation above):
\item $u(v_i) - u(b_{low}) \geq v_i[x(v_i) - x(b_{low}] - v_i
	\sum_{k=1}^{l}\text{jump in x}_i(\cdot, \bm{b_{-i}})$ 
\item $u(v_i) - u(b_{low}) \geq v_i[x(v_i) - x(b_{low})] - v_i [x(v_i) -
	x(b_{low})]$
\item $u(v_i) - u(b_{low}) \geq 0$
\end{enumerate}

Thus we have shown for monotone and piecewise constant allocation rule x,
bidding truthfully produces a utility that is $\geq$ the utility for all
higher bids and all lower bids. The utility is thus maximized when bidding
$v_i$; therefore, we have created a DSIC mechanism.

\section*{Exercise 13}
Consider the following extension of the sponsored search setting discussed in lecture. Each bidder i now has
a publicly known quality $\beta_i$ (in addition to a private valuation vi per click).
As usual, each slot j has a CTR $\alpha_j$, and $\alpha_1 \geq\alpha_2 · · ·
\geq\alpha_k$.\\

We assume that if bidder i is placed in slot j, its probability of a click is
$\beta_i \alpha_j$
thus, bidder i derives value $v_i\beta_i\alpha_j$ from this outcome.
Describe the surplus-maximizing allocation rule in this generalized sponsored search setting. Argue that
this rule is monotone. Give an explicit formula for the per-click payment of each bidder that extends this
allocation rule to a DSIC mechanism.

This surplus maximizing allocation rule for this auction is to give to the
slot with the highest CTR, slot 1, to the highest bidder, the slot with second
highest CTR to the second highest bidder and so on until the jth slot is
filled.  All bidders lower than the jth highest recieve no slots.  Although
its true allocation for the highest j bidders is 1 (they each get a slot), the
CTR for those slots is monotonically increasing. The payment rule is given as:
\begin{equation}
	p(b_i, \bm{b_{-i}}) = \dfrac{1}{\alpha_j - \alpha_{j-1}}\sum_{j=1}^{l}z_j[\text{jump in CTR between
	$\alpha_j$ and$\alpha_{j-1}$]}
\end{equation}

\section*{Exercise 14}
Consider an arbitrary single-parameter environment, with feasible set X. The
surplus-maximizing allocation rule is $x(\bm{b}) = arg max(x_1,...,x_n)\in X
\sum_{i=1}^{n}b_ix_i$. Prove that this allocation rule is monotone.
[You should assume that ties are broken in a deterministic and consistent way, such as lexicographically.]

\begin{proof}
	Assume that the allocation rule $x(\bm{b})$ is not monotone. That means
	there exists two bids $b_i, b_j$ with $b_i \leq b_j$ for which the
	allocation of $b_i \le b_j$ (ie. $x(b_i, \bm{b_{-i}}) \le
	x(b_j,\bm{b_{-j}})$). But since $b_i$ is $\leq b_j$, that means if we
	swapped their allocations, then the overall surplus would be greater than
	the surplus achieved with $x(\bm{b})$, but this contradicts that we
	defined $x(\bm{b})$ as the surplus-maximizing allocation.
\end{proof}

\section*{Exercise 15}
\section*{Exercise 16}
Continuing the previous exercise, consider a 0-1 downward-closed
single-parameter environment. Suppose you are given a “black box” that can
compute the surplus-maximizing allocation rule x(b) for an arbitrary input b.
Explain how to compute the payments identified in the previous exercise by invoking this black box multiple times.


\begin{enumerate}%%[label = Step \arabic*:]
	\item Use the blackbox to calculate the x($\bm{b}$) for the current bids $\bm{b}$.
	\item Calculate the surplus for this allocation as $S(\bm{b}) = \sum_{j \in x(\bm{b})}v_j$.
	\item For every bidder i with allocation 1 in $x(\bm{b})$, remove them
		from the set of bidders to create $b_{-i}$ and use the black box to
		calculate the surplus-maximizing allocation $x(\bm{b_{-i}})$.  Then
		for every bidder i, calculate the surplus as $S(\bm{b_{-i}}=)sum_{j \in
		x(\bm{b_{-i}})} v_j$
	\item For every bidder i with allocation 1 in $x(\bm{b})$, we can then
		calculate their payment (i.e. the critical bid)
		as the difference between between the surplus given by $S(\bm{b_{-i}})-[S(\bm{b}) - v_i]$
\end{enumerate}

\section*{Exercise 17}
Review the knapsack problem.  DP solution that is NP-hard with running time
O(Wn) where W is the capacity of the knapsack and n is the number if items.

\section*{Exercise 18}
Prove that the Knapsack auction allocation rule induced by the greedy
$\dfrac{1}{2}$-approximation algorithm covered in lecture is monotone.

Allocation rule covered in lecture says sort the items by the ratio of their
$\frac{bid}{capacity}$ in descending order.  Continue to include items in the
feasible set until an item does not fit.  Then, do not allocate all remaining
items.  That means our allocation looks like a step function, with a sequence
of zeros followed by a sequence of ones: x = {0, 0, ...1,1,1}.  There cannot
be any items not allocated preceding items that are included in the
allocations when sorted by their $\frac{bid}{capacity}$ ratio.

Suppose our allocation rule above was not monotone.  That means there exist
two items i, j for which $\frac{b_i}{w_i} \leq \frac{v_j}{w_j}$ where i is
included in the allocation and j is not.  That contradicts the defition of our  
allocation rule because we allocate in order of $\frac{bid}{capacity}$ ratio.

\end{document}
