\documentclass{article}
\usepackage{graphicx,fancyhdr,amsmath,amssymb,amsthm,subfig,url,hyperref,bm}
\usepackage[margin=1in]{geometry}

%----------------------- Macros and Definitions --------------------------

%%% FILL THIS OUT
\newcommand{\studentname}{Garrett Rodrigues}
\newcommand{\exerciseset}{Exercise Set 1}
%%% END



\renewcommand{\theenumi}{\bf \Alph{enumi}}

%\theoremstyle{plain}
%\newtheorem{theorem}{Theorem}
%\newtheorem{lemma}[theorem]{Lemma}

\fancypagestyle{plain}{}
\pagestyle{fancy}
\fancyhf{}
\fancyhead[RO,LE]{\sffamily\bfseries\large Stanford University}
\fancyhead[LO,RE]{\sffamily\bfseries\large CS 364A Algorithmic Game Theory}
\fancyfoot[LO,RE]{\sffamily\bfseries\large \studentname}
\fancyfoot[RO,LE]{\sffamily\bfseries\thepage}
\renewcommand{\headrulewidth}{1pt}
\renewcommand{\footrulewidth}{1pt}

\graphicspath{{figures/}}

%-------------------------------- Title ----------------------------------

\title{CS364A \exerciseset}
\author{\studentname}

%--------------------------------- Text ----------------------------------

\begin{document}
\maketitle

\section*{Exercise 9}
Use Myerson’s Lemma to prove that the Vickrey auction is the unique single-item auction that is DSIC,
always awards the good to the highest bidder, and charges losers 0.

Recall the three components of Myserson's Lemma (for single parameter
environments).
\begin{enumerate}
\item An allocation rule x is implementable if and only if it is monotone
\item If x is monotone, then there is a unique payment rule such that the sealed-bid mechanism (x, p) is DSIC [assuming the normalization that bi = 0 implies pi(b) = 0]. 
\item The payment rule in (b) is given by an explicit formula 
\end{enumerate}

\begin{equation}
	p(b_i, \bm{b_{-i}}) = \sum_{j=1}^{l}\text{jump in x}_i(\cdot,
	\bm{b_{-i}})\text{at $z_j$}
\end{equation}

Also recall that a Vickrey auction is defined as a sealed-bid auction where
the highest bidder wins and pays a price equal to the second highest bid.

Therefore, by definition, the Vickrey awards the good to the highest bidder
By Myerson's lemma, we know that for single parameter
enviroments (e.g. this single item auction), if x is monotone, then it is
implementable and there is a unique payment rule such that the sealed bid
mechanism is DSIC. In the Vickrey auction, the allocation rule looks like a
step function, with all bidders getting 0 allocation until we reach the
second highest bidder, after which the allocation is 1. Therefore, the
allocation rule in a Vickrey auction is monotone.

Let $b_{high}$ be the highest bidder and $\bm{b_{-high}}$ be all other
bidders.  Given the payment rule outlined in Myserson's lemma, the payment for
$\bm{b_{-high}}$ is 0 because there are no jumps before the highest bidder
(the only jump in the allocation happens AT the second highest bid).  For
$b_{high}$, the payment rule gives the prices as $z_j$ multiplied by the jump
in x.  $z_j$ is equal to the second highest bid and the jump in x at the
second highest bid is from 0 to 1.  Therefore the payment for the highest
bidder is equal to the second highest bid, exactly as outlined in the Vickrey
auction. 
\section*{Exercise 10}
Use the “payment difference sandwich” in the proof of Myerson’s Lemma to prove that if an allocation rule is not monotone, then it is not implementable.

In the payment sandwich, we defined two bids y and z with y $<$ z.  For those
two bids, we were able to use the "swapping trick" to come up with the payment
sandwich.  If x is not monotone, that means for some value of y and z (with y
smaller than z), there is an allocation such that $x(y) > x(z)$,
which implies that $x(y) - x(z) > 0$.  So from the payment sandwich, we get
that $y[x(y) - x(z)] \geq p(y) - p(z) \geq z[x(y) - x(z)]$.  Since we know
that $x(y) - x(z)$ is positive, a non-monotone allocation rule implies that y
must be $\geq$ z, which contradicts our definition that y $<$ z.
\section*{Exercise 11}
\end{document}

