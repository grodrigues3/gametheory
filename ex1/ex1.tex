\documentclass{article}
\usepackage{graphicx,fancyhdr,amsmath,amssymb,amsthm,subfig,url,hyperref}
\usepackage[margin=1in]{geometry}

%----------------------- Macros and Definitions --------------------------

%%% FILL THIS OUT
\newcommand{\studentname}{Garrett Rodrigues}
\newcommand{\exerciseset}{Exercise Set 1}
%%% END



\renewcommand{\theenumi}{\bf \Alph{enumi}}

%\theoremstyle{plain}
%\newtheorem{theorem}{Theorem}
%\newtheorem{lemma}[theorem]{Lemma}

\fancypagestyle{plain}{}
\pagestyle{fancy}
\fancyhf{}
\fancyhead[RO,LE]{\sffamily\bfseries\large Stanford University}
\fancyhead[LO,RE]{\sffamily\bfseries\large CS 364A Algorithmic Game Theory}
\fancyfoot[LO,RE]{\sffamily\bfseries\large \studentname}
\fancyfoot[RO,LE]{\sffamily\bfseries\thepage}
\renewcommand{\headrulewidth}{1pt}
\renewcommand{\footrulewidth}{1pt}

\graphicspath{{figures/}}

%-------------------------------- Title ----------------------------------

\title{CS364A \exerciseset}
\author{\studentname}

%--------------------------------- Text ----------------------------------

\begin{document}
\maketitle

\section*{Problem 1}
The badminton tournament from the 2012 London Olympics could be redesigned to
prevent intentional losing in the following ways:

\begin{enumerate}
\item Getting rid of a group stage and making a pure tournament, where any
	loss at any time removes you from the tournament 
\item Giving a "bye" rounds to the teams that finished highest in the
		group stages, thereby incentivizing always winning
\item Creating a ranked ladder that based on win-loss and point differential, where every team players every other team (insertion sort type approach)
\end{enumerate}


\section*{Problem 2}
\begin{enumerate}
\item %A
	The scene from a \textbf{A Beautiful Mind} doesn't show a true Nash Equilibrium,
		because with a true Nash Equilibrium, no player should benefit from
		changing their strategy, assuming the other players hold their stategy
		constant. In the movie, the four men have the
		option to pursue 5 women, and Nash suggests that the
		best strategy is to NOT pursue the most attractive woman, "The
		Blonde."; them motivation for this approach is that the four guys will
		not "get in each others' way." This is not a Nash equilibrium, because if each
		player knew the others strategy, that is, they knew the other men
		would not be pursuing the Blonde, nothing would prevent any one of them
		from alerting their strategy and pursuing the Blonde (which
		effectively increases their utility).

\end{enumerate}

\section*{Problem 3}
Prove that there is a unique (mixed-strategy) Nash equilibrium in the Rock-Paper-Scissors game described in class.

Recall the definition of a NE.  There is exists a stratey for each player in
		the game such that it does not improve their utility to change
		strategies when their opponents strategies are fixed.

Setup.  
\begin{enumerate}
\item Two players p and q
\item pr, pp, ps = probabilility player p plays rock, paper scissors
	respectively.  Since it's a symmetric game, the probabilities for player q
		should be the same

\item$\mathbb{E}(rock) = 1*ps - 1*pp$
\item$\mathbb{E}(scissors) = 1*pp - 1*pr$
\item$\mathbb{E}(paper) = 1*pr - 1*ps$
\item$\mathbb{E}(rock) = \mathbb{E}(scissors) = \mathbb{E}(paper)$
\item pr + pp + ps = 1
\end{enumerate}

Then, since we've reached an Equilibrium, the payoffs should be stable meaning
it would not be beneficial for any player to change from one option to
another, so the expected value of each of options should be equal.  With a
little algebra we can show that the NE strategy for both players is to play
each option with probabilility $1/3$.  That is, $pp = pr = ps = 1/3$.

\section*{Problem 4}
Compare and contrast an eBay auction with the sealed-bid second-price auction described in class. (Read up on eBay auctions if you don’t already know how they work.) Should you bid differently in the two auctions?

\section*{Problem 5}
Consider a single-item auction with at least three bidders. Prove that awarding the item to the highest bidder, at a price equal to the third-highest bid, yields an auction that is not dominant-strategy incentive
compatible (DSIC).

Recall definition of DSIC - it means the strategy that maximizes utility for
every bidder is bidding their true valuations, and they would continue to do
that regardless of what the other bidders do.  Without loss of generality,
consider that n bidders, with n>= 3,  are ranked from 1 - n according to their
nbids $b\textsubscript{1}, b_2, ... b_n$. In a DSIC auction, the bidders bids
$b_i$ would match their valuations $v_i$ in order to maximize each of their
individual utility.  

In this auction, the utility for every bidder except the highest bidder is 0
(they don't win the item), and the utility for the highest bidder is $v_n$ -
$b_{n-2}$.

For every bidder with valuation up to and including $v_{n-2}$, reducing their
bid has no impact and increasing their bid could only result in negative
utility since winning the item at a price higher than their valuation results
in negative utility. However, the bidder with the second highest valuation,
is incentivized to increase its bid above the highest bidder. If that
happens, the bidder with the second highest valuation will win the auction and
still pay a price lower than his valuation, increasing his utility above 0. In
response, the highest bidder can further increase his bid so to win again.
The resulting outcome is a seesaw between the top two bidders who are
incentivized to overbid and outbid eachother perpetually in order to win the
auction while achieving positive utility.

\section*{Problem 6}

Suppose there are k identical copies of a good and n > k bidders. Suppose also that each bidder can receive
at most one good. What is the analog of the second-price auction? Prove that your auction is DSIC.

An analogous auction is one that where the bidders are sorted from highest to
lowest and the top k bidders receive the k items.  All of them pay the same
price which is \textbf{equal to the k+1th person's bid}. In this case, the surplus is
maximized, because for each bidder, their utility is maximized. There are two
cases to consider:
\begin{enumerate} 
\item The top k bidders. None of the top k bidders would benefit from changing their bids (raising their bids
doesn't get them more of the good and lowering their bid doesn't impact
the price unless they bid below the k+1th bid, in which case they no longer
get an item). 
\item The bottom k+1, k+2 ... n bidders.  By raising their bids above their
	true valuations (overbidding) of them would achieve negative utility (p
		would be greater than v), so it is better to not win the item and
		maintaing 0 utility.
		
\end{enumerate}

An alternative auction that appears "awesome", but is not DSIC is one where
the top k bidders each pay the bid of the bidder directly below them.  This is not DSIC, because DSIC implies that for all players bidding truthfully
maximizes their utility.  If utility is value - price, the top k-1 bidders
could try to underbid their true value so that they got the kth spot, paying
the lowest possible price and still winning the item.

\section*{Problem 7}
Suppose you want to hire a contractor to perform some task, like remodeling a house. Each contractor has a private cost for performing the task. Give an analog of the Vickrey auction in which contractors report their costs and the auction chooses a contractor and a payment. Truthful reporting should be a dominant strategy in your auction and, assuming truthful bids, your auction should select the contractor with the smallest private cost.
[Aside: auctions of this type are called procurement or reverse auctions.]

An analogy for a Vickrey auction in this case would be one where the
contractor with the lowest reported costs wins the contract and is paid the
price of the second lowest contractors reported costs.  The utility for each
contractor that doesn't win the contract is 0.  And the utility for the
winning contractor is the $Payment - TrueCost$.  If the contractors are sorted
according to their true cost for lowest to highest, then no contractor is
incentivizied to change their $ReportedCost$.  For the n-1 contractors that
lost the auction, increasing their $ReportedCost$ would not change their
utility; it would still be zero, and decreasing their $ReportedCost$ could
potentially win them the auction, but the payment received would always be
less than or equal to their $TrueCost$, so that their utility would go from 0
to negative.  The winning bidder has no incentive to change their bid at all,
since their bid is not reflected in the Payment by lowering their bid, they
risk losing the auction and dropping their utility from positive to 0.

\end{document}
